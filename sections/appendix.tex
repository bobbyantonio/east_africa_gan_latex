\documentclass[../main.tex]{subfiles}
\graphicspath{{\subfix{../images/}}}


\begin{document}

\section{Forecast variables used}\label{app:fcst_vars}
The IFS variables used to train the model are shown in Table~\ref{tab:vars}, definitions taken from~\citep{ecmwf_parameter_2023}.

The preprocessing methods mentioned in the table are as follows, using the year 2017 as the reference period:
\begin{itemize}
    \item Minmax: calculate the minimum $d_{\text{min}}$ and maximum $d_{\text{max}}$ over the reference period, and then transform each value $v$ according to $(v - d_{\text{min}}) / (d_{\text{max}} - d_{\text{min}})$.
    \item Max: calculate the and maximum $d_{\text{max}}$ over the reference period, and then transform each value $v$ according to $v  / d_{\text{max}}$.
    \item Log: Transform each value $v$ according to $\log_{10}(1+v)$.
\end{itemize}
\begin{table}[ht!]
\centering
\begin{tabular}{c | c | c } 
 \hline
 Variable name & Symbol & Pre-processing applied \\ [0.5ex] 
 \hline\hline
 2 metre temperature &2t & Minmax  \\
 Convective available potential energy &cape & Log \\
 Convective inhibition &cin & Max \\
Convective precipitation &cp & Log \\
Surface pressure & sp & Minmax  \\
Total column cloud liquid water &tclw & Max \\
Total column vertically-integrated water vapour&tcwv & Log \\
Top of atmosphere incident solar radiation&tisr & Max \\
Total precipitation &tp & Log \\
Relative humidity at 200hPa  &r200 & Max \\
Relative humidity at 700hPa  &r700 & Max \\
Relative humidity at 950hPa  &r950 & Max \\
Temperature at 200hPa &t200 & Minmax \\
Temperature at 700hPa  &t700 & Minmax \\
Eastward component of wind at 200hPa &u200 & Max \\
Eastward component of wind at 700hPa &u700 & Max \\
Northward component of wind at 200hPa&v200 & Max \\
Northward component of wind at 700hPa &v700 & Max \\
Vertical velocity at 200hPa &w200 & Max \\
Vertical velocity at 500hPa &w500 & Max \\
Vertical velocity at 700hPa &w700 & Max \\
 \hline
\end{tabular}

\caption{IFS variables used to train the model, as well as the normalisation applied to each variable. See text for description of the different preprocessing types.}
\label{tab:vars}
\end{table}

% \section{Rainfall by month}

% In Fig.~\ref{fig:monthly_rain} we plot the IMERG monthly rainfall from 2001-2020.

% \begin{figure}[b]
%      \centering
%      \includegraphics[width=0.8\textwidth]{month_averages.pdf}
     
%      \caption{Average monthly rainfall over the period 2001-2020.}
%      \label{fig:monthly_rain}
% \end{figure}

\ifSubfilesClassLoaded{%
    \bibliographystyle{alpha}
    \bibliography{references_z}

}{}


\end{document}